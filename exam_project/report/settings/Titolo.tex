% ------------------------------------------------------------------

\thispagestyle{empty}                                                 
\begin{center}                                                            
    \vspace{5mm}
    {\LARGE UNIVERSITÀ DI BOLOGNA} \\                       
      \vspace{5mm}
\end{center}
\begin{center}
  \includegraphics[scale=.27]{img/0-Front-page/logo_unibo}
\end{center}
\begin{center}
      \vspace{5mm}
      {\LARGE School of Engineering} \\
        \vspace{3mm}
      {\Large Master Degree in Automation Engineering} \\
      \vspace{20mm}
      {\LARGE Distributed Autonomus System} \\
      \vspace{5mm}{\Large\textbf{Course Project}}                  
      \vspace{10mm}
\end{center}
\begin{flushleft}                                                                              
     {\large Professor:} \\
       \vspace{0.3 cm}
       \large{{\textbf{Giuseppe Notarstefano}}}

       \vspace{0.1 cm}
       \large{{\textbf{Ivano Notarnicola}}}       
     \vspace{13mm}
\end{flushleft}
\begin{flushright}
      {\large Students:}\\
        \vspace{0.3 cm}
        \large{{\textbf{Andrea Alboni}}}

        \vspace{0.1 cm}
        {\large{{\textbf{Federico Calzoni}}}}

        \vspace{0.1 cm}
        {\large{{\textbf{Emanuele Monsellato}}}}
\end{flushright}        %capoverso allineato a destra
\begin{center}
\vfill
      {\large Academic year \@2024/2025} \\
\end{center}


\newpage
\thispagestyle{empty}

%%%%%%%%%%% Abstract %%%%%%%%%%%%
\begin{center}
\chapter*{}
\thispagestyle{empty}
{\Huge \textbf{Abstract}}\\
\vspace{15mm}
\end{center}

% \listoffigures\thispagestyle{empty}
A key challenge in the optimal control of robotic systems is the generation of optimal trajectories while considering the system's dynamics and constraints. This project focuses on the design and implementation of an optimal control law for a flexible robotic arm, modeled as a planar two-link robot with torque applied to the first joint. 

The setup phase involves discretizing the robot's dynamics and formulating the discrete-time state-space equations.

The first task focuses on generating an optimal trajectory between two equilibrium points of the robotic arm. The equilibrium states are computed using a root-finding algorithm and a symmetric reference trajectory is defined between them. 

The second task extends the first by introducing a smooth desired trajectory, for which a quasi-static trajectory is computed as an initial guess. This trajectory is refined through optimal control methods to minimize the cost function.

The third task involves trajectory tracking using Linear Quadratic Regulator (LQR) control. By linearizing the system dynamics around the optimal trajectory, a closed-loop controller is designed to track the reference trajectory while compensating for initial perturbations.

The fourth task leverages Model Predictive Control (MPC) to achieve trajectory tracking. The controller is developed to handle constraints dynamically, showcasing robustness to perturbed initial conditions.

Finally, the fifth task includes the visualization of the robotic arm's motion by animating the results of the LQR trajectory tracking task.