\usepackage{amsfonts}
\usepackage{amssymb,amsmath,color}


% character encoding luaLatex
\usepackage{fontspec}

\usepackage[hidelinks]{hyperref}
\usepackage{comment}

\usepackage[linesnumbered]{algorithm2e}
\RestyleAlgo{ruled}

\usepackage{amssymb}

\usepackage{titlesec} % appendix title size

%%% for math systems with confitions
\usepackage{amsmath}
%\[
%    f(x)= 
%\begin{cases}
%    \frac{x^2-x}{x},& \text{if } x\geq 1\\
%    0,              & \text{otherwise}
%\end{cases}
%\]
\usepackage{nicematrix}

%%% for tables
\usepackage{multirow}
\usepackage{longtable} %tabelle multipagina
\usepackage{booktabs}

%%% IMAGES
\usepackage{graphicx}
\usepackage{svg}
\usepackage[export]{adjustbox}
\usepackage{subcaption}
\usepackage{float}
%permette rotazione box
\usepackage{rotating}
\usepackage{tikz}
\usepackage{tcolorbox}


%% colours of links and index
\begin{comment}
\hypersetup{
  colorlinks   = true, %Colours links instead of ugly boxes
  urlcolor     = blue, %Colour for external hyperlinks
  linkcolor    = blue, %Colour of internal links
  citecolor   = blue %Colour of citations, could be ``red''
}
\end{comment}
\graphicspath{ {./img/} }

%%% FOR CODE
\usepackage{listings}

\colorlet{mycoolgray}{gray!40}
\lstdefinestyle{output}{
	numbers=none, % where to put the line-numbers
	numberstyle=\tiny, % the size of the fonts that are used for the line-numbers     
	backgroundcolor=\color{darkgray},
	basicstyle=\ttfamily\color{white},
	captionpos=b, % sets the caption-position to bottom
	breaklines=true, % sets automatic line breaking
	breakatwhitespace=false, 
}
\lstdefinestyle{cmd}{
	numbers=none, % where to put the line-numbers
	numberstyle=\tiny, % the size of the fonts that are used for the line-numbers     
	backgroundcolor=\color{mycoolgray},
	basicstyle=\ttfamily\color{black},
	captionpos=b, % sets the caption-position to bottom
	breaklines=true, % sets automatic line breaking
	breakatwhitespace=false, 
}

\usepackage{color} %red, green, blue, yellow, cyan, magenta, black, white
\definecolor{mygreen}{RGB}{28,172,0} % color values Red, Green, Blue
\definecolor{mylilas}{RGB}{170,55,241}


\lstset{language=Matlab,%
    %basicstyle=\color{red},
    breaklines=true,%
    morekeywords={matlab2tikz},
    keywordstyle=\color{blue},%
    morekeywords=[2]{1}, keywordstyle=[2]{\color{black}},
    identifierstyle=\color{black},%
    stringstyle=\color{mylilas},
    commentstyle=\color{mygreen},%
    showstringspaces=false,%without this there will be a symbol in the places where there is a space
    numbers=left,%
    numberstyle={\tiny \color{black}},% size of the numbers
    numbersep=7pt, % this defines how far the numbers are from the text
    emph=[1]{for,end,break},emphstyle=[1]\color{red}, %some words to emphasise
    %emph=[2]{word1,word2}, emphstyle=[2]{style},    
}

\usepackage{pythonhighlight} % python code
% usage:
%\begin{python}
% code...
%\end{python}

% you can load an external Python file with:
%\inputpythonfile{python_file.py}[23][50]
% display the contents from line 23 to line 50

%% for todo notes
\setlength {\marginparwidth }{2cm} 

%\usepackage[colorinlistoftodos,prependcaption,textsize=tiny]{todonotes}
\usepackage{todonotes}
%how to put todo
%\todo{inserire fonte mypersonal trainer}
%\todo[inline]{inserire fonte mypersonal trainer}


% %Per realizzare sistemi di equazioni 
% \newenvironment{system}%
% {\left\lbrace\begin{array}{@{}l@{}}}%
% {\end{array}\right.}

% \begin{equation}
% \begin{system}
% equation 1  \\
% equation 2 
% \end{system}
% \end{equation}

\newlength{\maxfigurewidth}

\sloppy % to avoid overfull warnings