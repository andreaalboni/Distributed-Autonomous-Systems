\chapter{Introduction}
% \addcontentsline{toc}{chapter}{Introduction}

Distributed Autonomous Systems (DAS) are becoming increasingly relevant in a variety of application domains, ranging from environmental 
monitoring to autonomous transportation and surveillance. These systems rely on the coordination and cooperation of multiple agents such as
mobile robots or sensors that operate autonomously, but are connected through a communication network. The decentralized nature of DAS enables 
scalability, fault tolerance, and adaptability in complex environments


\noindent
This report presents the development and simulation of algorithms that enable such systems to perform complex tasks in a distributed manner. 
In particular, the work is structured around two main tasks: 
\begin{itemize}
    \item \textbf{Task 1:} \textit{Multi-Robot Target Localization}
    \item \textbf{Task 2:} \textit{Aggregative Optimization for Multi-Robot Systems}
\end{itemize}

\noindent In the first task addresses the problem where a fleet of robots aims to cooperatively estimate the positions of unknown targets based on
noisy local measurements. This task emphasizes the use of distributed consensus optimization techniques, with a focus on the Gradient Tracking 
algorithm, which allows each robot to iteratively refine its estimate by combining local information with data received from neighboring agents.


\noindent The second task explores a distributed optimization framework where each robot aims to minimize a local objective function 
that depends not only on its own state but also on a global aggregative quantity, typically the team's barycenter. 
The primary goal is to design and implement a distributed control algorithm that enables the robots to stay close to private targets while 
maintaining the cohesion of the fleet, despite communication constraints. This task employs the Aggregative Tracking algorithm to achieve 
coordination and consensus through purely local interactions, relying solely on local communications. The Aggregative Tracking algorithm is 
employed to solve this problem in a distributed fashion, and the strategy is further validated through a ROS 2 implementation.


The purpose of this project is to apply theoretical concepts learned during the DAS course to practical scenarios through the implementation of 
distributed algorithms, simulation of multi-agent systems, with a focus on convergence properties and performance evaluation.